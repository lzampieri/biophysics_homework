\documentclass[11pt,a4 paper]{article}
\usepackage{amsmath, amsthm} 
\usepackage[english]{babel}
\usepackage[T1]{fontenc}
\usepackage[utf8]{inputenc}
\usepackage[margin=2cm]{geometry}
\usepackage{indentfirst}
\usepackage{graphicx}
\usepackage{subfigure}
\usepackage{caption}
\usepackage{siunitx}
\captionsetup{tableposition=top,font=small,width=0.8\textwidth}
\usepackage{booktabs}
\usepackage[arrowdel]{physics}
\usepackage{mathtools}
\usepackage{tablefootnote}
\usepackage{amssymb}
\usepackage{enumitem}
\usepackage{multicol}
\usepackage{hyperref}
\setlist[description]{font={\scshape}} %style=unboxed,style=nextline
\usepackage{wrapfig}
\usepackage{float}
\usepackage{floatflt}
\usepackage{commath}
\usepackage{bm}
\usepackage{nicefrac}
\usepackage{xspace}
\usepackage{ifthen}
\usepackage{comment}
\usepackage[table]{xcolor}
\usepackage[colorinlistoftodos,textsize=tiny]{todonotes}
% \usepackage[autostyle,italian=guillemets]{csquotes}
% \usepackage[backend=biber,style=alphabetic,maxalphanames=4,maxbibnames=6]{biblatex}
% \addbibresource{D:/ZoteroBib.bib}
% \addbibresource{D:/ZoteroSGSSnatBib.bib}


\usepackage{chemformula}

\renewcommand*{\thefootnote}{\fnsymbol{footnote}}
\sisetup{exponent-product = \cdot}
\newcommand{\tc}{\,\mbox{tc}\,}
\newcommand{\Epsilon}{\mathcal{E}}
\renewcommand*{\epsilon}{\varepsilon}
\newcommand{\half}{\frac{1}{2}}
\renewcommand{\ev}[1]{\eval{}_{#1}}
\newcommand{\overbar}[1]{\mkern 1.5mu\overline{\mkern-1.5mu#1\mkern-1.5mu}\mkern 1.5mu}
\renewcommand{\underbar}[1]{\mkern 1.5mu\underline{\mkern-1.5mu#1\mkern-1.5mu}\mkern 1.5mu}
\let\oldfrac\frac
\renewcommand{\frac}[3][d]{\ifthenelse{\equal{#1}{d}}{\oldfrac{#2}{#3}}{\nicefrac{#2}{#3}}}
\newcommand{\fourier}{\mathcal{F}}
\DeclareMathOperator{\arcsinh}{arcsinh}
\DeclareMathOperator{\const}{const}
\let\var\undefined
\DeclareMathOperator{\var}{var}
\DeclareMathOperator{\erfc}{erfc}
\newcommand\numberthis{\addtocounter{equation}{1}\tag{\theequation}}

\setlist[itemize]{noitemsep}

\title{To do}
\author{L. Zampieri - mat. 1237351\\Matlab exercise for Biological Physics exam}
\date{\today}

\begin{document}
    
\maketitle

\section*{Abstract}
In a living cell, major transmembrane currents components are given by sodium and potassium ions. While the extracellular fluid can be considered a infinite reservoir of ions, and their concentration can be considered constant, the cytoplasm have a finite volume and the concentration of the sodium and potassium ions inside the cell is strongly affected by transmembrane currents. In absence of active sodium and potassium pumps, deputed to maintain the correct concentration gradient between the two sides of the cellular membrane, the ions flux will depolarize the cell leading to a null resting potential difference and a null resting concentration gradient. In this paper, the Goldman-Hodgkin-Katz current equation\todo{Citazione necessaria} will be used to model the membrane of a non-excitable cell, i.e. a cell without voltage-gated sodium and potassium channels, in absence of active pumps. A typical mammalian cell will be numerical simulated and the results discussed.

\section{Introduction}
The physical parameters of a living cell are fundamental to keep functional all the cell apparatus, and to permit an efficient carrying out of the chemical and physical process of the cell lifecycle. Among them, the concentration of ions, and in particular of potassium and sodium one, and the resting potential. They are fundamental, for example, to the correct operation of complex transmembrane proteins, and in general to keep the cell alive. Given the concentration of a ion specie inside the cell $[S]_{in}$ and the concentration of the same specie outside the cell $[S]_{out}$, the Nerst potential is the hypothetical transmembrane potential which completely counterbalance the effects of concentration gradients, leading to a stable equilibrium. It can be computed as:
\begin{align*}
    V_N = \frac{RT}{zF} \ln\frac{[S]_{out}}{[S]_{in}}
\end{align*}

The resting cell potential is typically different from the Nerst one, and therefore the ions concentrations are not at equilibrium. This means that a current flow between the cell membrane, moving ions between the two sides. Typically, these current are counterbalanced by active pumps, which uses ATP to transport ions counter-gradient: for example, the \ch{Na+ /K+}-ATPase. If these pumps are not present, the total ions fluxes are not null and the cells stability is lost.

The extracellular liquid can be considered infinitely-spread and, in our simplified model, homogeneous: it acts like a reservoir of ions, and therefore the variations in concentrations induced by ions fluxes the cell are negligible. On the other side, the cell have a finite volume and therefore a finite number of ions inside it: the ions fluxes strongly affect the concentration inside the cell, and can lead to a depolarization of the membrane.

\section{Model}

\end{document}